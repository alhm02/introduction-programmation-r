\documentclass[landscape]{vgslides}
  \usepackage{vgmath,actu,amsmath,amsthm,url}
  \usepackage{palatino,mathpazo}
  \usepackage{textcomp,fourier-orns,mathabx}
  \usepackage{Sweave-sans-ae}
  \usepackage{threeparttable,paralist,expdlist}

  %%% Utilisation des num�ros de chapitres
  \newcounter{chapter}
  \makeatletter
  \renewcommand\thesection{\thechapter.\@arabic\c@section}
  \renewcommand\thesubsection{\thesection.\@arabic\c@subsection}
  \renewcommand\thesubsubsection{\thesubsection .\@arabic\c@subsubsection}
  \makeatother

  %%% D�sactivation de certaines commandes
  \newcommand{\R}{}
  \newcommand{\Splus}{}
  \newcommand{\Index}[1]{}
  \newcommand{\indexargument}[1]{}
  \newcommand{\Indexargument}[1]{}
  \newcommand{\indexattribut}[1]{}
  \newcommand{\Indexattribut}[1]{}
  \newcommand{\indexclasse}[1]{}
  \newcommand{\Indexclasse}[1]{}
  \newcommand{\indexfonction}[1]{}
  \newcommand{\Indexfonction}[1]{}
  \newcommand{\indexmode}[1]{}
  \newcommand{\Indexmode}[1]{}
  \newcommand{\indexobjet}[1]{}
  \newcommand{\Indexobjet}[1]{} 
  \newcommand{\indexemacs}[1]{}
  \newcommand{\indexess}[1]{}
  
  %%% Styles pour les environnements d'exemples et al.
  \theoremstyle{plain}
  \newtheorem{ex}{Exemple}[chapter]
  \theoremstyle{remark}
  \newtheorem*{rem}{Remarque}
  \theoremstyle{definition}
  \newtheorem*{astuce}{Astuce}
  \newenvironment{sol}{\begin{proof}[Solution]}{\end{proof}}

  %%% Environnement pour les listes de commandes
  \newenvironment{ttscript}[1]{%
    \begin{list}{}{%
        \setlength{\labelsep}{1.5ex}
        \settowidth{\labelwidth}{\code{#1}}
        \setlength{\leftmargin}{\labelwidth}
        \addtolength{\leftmargin}{\labelsep}
        \setlength{\parsep}{0.5ex plus0.2ex minus0.2ex}
        \setlength{\itemsep}{0.3ex}
        \renewcommand{\makelabel}[1]{##1\hfill}}}
    {\end{list}}

  %%% Environnement pour les listes de structures de contr�le
  \newenvironment{struclist}{%
    \begin{description}[\breaklabel\setlabelstyle{\mdseries\ttfamily}%
      \setleftmargin{\parindent}]}
    {\end{description}}

  %%% Styles pour les noms de fonctions, code, etc.
  \newcommand{\code}[1]{\texttt{#1}}
  \newcommand{\attribut}[1]{\code{#1}}
  \newcommand{\Attribut}[1]{\code{#1}}
  \newcommand{\argument}[1]{\code{#1}}
  \newcommand{\Argument}[1]{\code{#1}}
  \newcommand{\classe}[1]{\code{#1}}
  \newcommand{\Classe}[1]{\code{#1}}
  \newcommand{\fonction}[1]{\code{#1}}
  \newcommand{\Fonction}[1]{\code{#1}}
  \newcommand{\mode}[1]{\code{#1}}
  \newcommand{\Mode}[1]{\code{#1}}
  \newcommand{\objet}[1]{\code{#1}}
  \newcommand{\Objet}[1]{\code{#1}}
  \newcommand{\emacs}[1]{\code{#1}}
  \newcommand{\ess}[1]{\code{#1}}

\begin{document}

\renewcommand{\FrenchLabelItem}{\small$\sqbullet$}
\renewcommand{\labelitemii}{\textendash}

\setcounter{chapter}{1}

\begin{slide}
  \begin{center}
    \LARGE\bfseries
    PR�SENTATION DU LANGAGE S
  \end{center}
\end{slide}


\begin{slide}
  \section{Le langage S}
  \label{presentation:langage}

  Le S est un langage pour �programmer avec des donn�es� d�velopp�
  chez Bell Laboratories (anciennement propri�t� de AT\&T, maintenant
  de Lucent Technologies).

  \begin{itemize}
  \item Ce n'est pas seulement un �autre� environnement statistique
    (comme SPSS ou SAS, par exemple), mais bien un langage de
    programmation complet et autonome.
  \item Le S est inspir� de plusieurs langages, dont l'APL et le Lisp:
    \begin{itemize}
    \item interpr�t� (et non compil�);
    \item sans d�claration obligatoire des variables;
    \item bas� sur la notion de vecteur;
    \item particuli�rement puissant pour les applications
      math�matiques et statistiques (et donc actuarielles).
    \end{itemize}
  \end{itemize}
\end{slide}

\begin{slide}
  \section{Les moteurs S}
  \label{presentation:moteurs}

  Il existe quelques �moteurs� ou dialectes du langage S.

  \begin{itemize}
  \item Le plus connu est S-Plus, un logiciel commercial de Insightful
    Corporation. (Bell Labs octroie � Insightful la licence exclusive
    de leur syst�me S.)
  \item \textsf{R}, ou GNU S, est une version libre (\emph{Open
      Source}) �\emph{not unlike S}�.
  \end{itemize}

  S-Plus et \textsf{R} constituent tous deux des environnements
  int�gr�s de manipulation de donn�es, de calcul et de pr�paration de
  graphiques.
\end{slide}

\begin{slide}
  \section{O� trouver de la documentation}
  \label{presentation:doc}

  S-Plus est livr� avec quatre livres, mais aucun ne s'av�re vraiment
  utile pour apprendre le langage S.

  Plusieurs livres --- en versions papier ou �lectronique, gratuits ou
  non --- ont �t� publi�s sur S-Plus et/ou \textsf{R}. On trouvera des
  listes exhaustives dans les sites de Insightful et du projet
  \textsf{R}:
  \begin{itemize}
  \item \url{http://www.insightful.com/support/splusbooks.asp}
  \item \url{http://www.r-project.org} (dans la section
    \texttt{Documentation}).
  \end{itemize}
\end{slide}

\begin{slide}
  \section{Interfaces pour S-Plus et \textsf{R}}
  \label{presentation:interfaces}

  Provenant du monde Unix, tant S-Plus que \textsf{R} sont d'abord et
  avant tout des applications en ligne de commande (\texttt{sqpe.exe}
  et \texttt{rterm.exe} sous Windows).

  \begin{itemize}
  \item S-Plus poss�de toutefois une interface graphique �labor�e
    permettant d'utiliser le logiciel sans trop conna�tre le langage
    de programmation.
  \item \textsf{R} dispose �galement d'une interface graphique
    rudimentaire sous Windows et Mac OS.
  \item L'�dition s�rieuse de code S b�n�ficie cependant grandement
    d'un bon �diteur de texte.
  \item � la question 6.2 de la foire aux questions (FAQ) de
    \textsf{R}, �Devrais-je utiliser \textsf{R} � l'int�rieur de
    Emacs?�\index{Emacs}, la r�ponse est: �Oui, d�finitivement.�
  \item Nous partageons cet avis, aussi apprendra-t-on � utiliser
    S-Plus ou \textsf{R} � l'int�rieur de GNU Emacs avec le mode
    ESS\index{ESS}.
  \item Autre option: WinEdt (partagiciel) avec l'ajout R-WinEdt.
  \end{itemize}
\end{slide}

\begin{slide}
  \section{Installation de Emacs avec ESS}
  \label{presentation:emacs}

  Il n'existe pas de proc�dure d'installation similaire aux autres
  applications Windows pour Emacs.  L'installation n'en demeure pas
  moins tr�s simple: il suffit de d�compresser un ensemble de fichiers
  au bon endroit.

  \begin{itemize}
  \item Pour une installation simplifi�e de Emacs et ESS, consulter le
    site Internet
    \begin{quote}
      \url{http://vgoulet.act.ulaval.ca/pub/emacs/}
    \end{quote}
    On y trouve une version modifi�e de GNU Emacs et des instructions
    d'installation d�taill�es.
  \item L'annexe A du document pr�sente les plus importantes commandes
    � conna�tre pour utiliser Emacs et le mode ESS.
  \end{itemize}
\end{slide}

\begin{slide}
  \section{D�marrer et quitter S-Plus ou \textsf{R}}
  \label{presentation:demarrer}

  \begin{itemize}
  \item Pour d�marrer \textsf{R} \R � l'int�rieur de Emacs:

\begin{verbatim}
M-x R RET
\end{verbatim}

    puis sp�cifier un dossier de travail . Une console \textsf{R} est
    ouverte dans un \emph{buffer} nomm� \texttt{*R*}.
  \item Pour d�marrer S-Plus sous Windows, consulter l'annexe B du
    document.
  \item Pour quitter, deux options sont disponibles:
    \begin{enumerate}
    \item Taper \fonction{q()} � la ligne de commande.
    \item Dans Emacs, faire \ess{C-c C-q}. ESS va alors s'occuper de
      fermer le processus S ainsi que tous les \emph{buffers} associ�s
      � ce processus.
    \end{enumerate}
  \end{itemize}
\end{slide}

\begin{slide}
  \section{Strat�gies de travail}
\label{presentation:strategies}

Il existe principalement deux fa�ons de travailler avec S-Plus et
\textsf{R}.
\begin{enumerate}
\item Le code est virtuel et les objets sont r�els. 

  C'est l'approche qu'encouragent les interfaces graphiques, mais
  aussi la moins pratique � long terme. 

  On entre des expressions directement � la ligne de commande pour les
  �valuer imm�diatement.

  Les objets cr��s au cours d'une session de travail sont sauvegard�s.

  Par contre, le code utilis� pour cr�er ces objets est perdu lorsque
  l'on quitte S-Plus ou \textsf{R}, � moins de sauvegarder celui-ci dans
  des fichiers.
  \newpage
\item Le code est r�el et les objets sont virtuels. 

  C'est l'approche que nous favoriserons. 

  Le travail se fait essentiellement dans des fichiers de script (de
  simples fichiers de texte) dans lesquels sont sauvegard�es les
  expressions (parfois complexes!) et le code des fonctions
  personnelles. 

  Les objets sont cr��s au besoin en ex�cutant le code.  
  \newpage
  Emacs permet ici de passer efficacement des fichiers de script �
  l'ex�cution du code:
  \begin{enumerate}[i)]
  \item D�marrer un processus S-Plus (\texttt{M-x Sqpe}) ou
    \textsf{R} (\texttt{M-x R}) et sp�cifier le dossier de travail.
  \item Ouvrir un fichier de script avec \ess{C-x C-f}. Pour cr�er un
    nouveau fichier, ouvrir un fichier n'existant pas.
  \item Positionner le curseur sur une expression et faire \ess{C-c
      C-n} pour l'�valuer.
  \item Le r�sultat appara�t dans le \emph{buffer} \texttt{*S+6*} ou
    \texttt{*R*}.
  \end{enumerate}
\end{enumerate}
\end{slide}


\begin{slide}
  \section{Gestion des projets ou environnements de travail}
  \label{presentation:workspace}

  S-Plus et \textsf{R} ont une mani�re diff�rente, mais tout aussi
  particuli�re de sauvegarder les objets cr��s au cours d'une session
  de travail.
  \begin{itemize}
  \item Tous deux doivent travailler dans un dossier et non avec des
    fichiers individuels.
  \item Dans S-Plus, \Splus tout objet cr�� au cours d'une session de
    travail est sauvegard� de fa�on permanente sur le disque dur dans
    le sous-dossier \texttt{\_\_Data} du dossier de travail.
  \item Dans \textsf{R}, \R les objets cr��s sont conserv�s en m�moire
    jusqu'� ce que l'on quitte l'application ou que l'on enregistre le
    travail avec la commande \fonction{save.image()}. L'environnement
    de travail (\emph{workspace}) est alors sauvegard� dans le fichier
    \texttt{.RData} dans le dossier de travail.
  \end{itemize}
\end{slide}

\begin{slide}
  Le dossier de travail est d�termin� au lancement de l'application.
  \begin{itemize}
  \item Avec Emacs et ESS on doit sp�cifier le dossier de travail �
    chaque fois que l'on d�marre un processus S-Plus ou R.
  \item Les interfaces graphiques permettent �galement de sp�cifier le
    dossier de travail.
    \begin{itemize}
      \sloppy
    \item Dans \Splus l'interface graphique de S-Plus, choisir
      \texttt{General Settings} dans le menu \texttt{Options}, puis
      l'onglet \texttt{Startup}. Cocher la case \texttt{Prompt for
        project folder}.  Consulter �galement le chapitre 13 du guide
      de l'utilisateur de S-Plus.
    \item Dans \R l'interface graphique de \textsf{R}, le plus simple
      consiste � changer le dossier de travail � partir du menu
      \texttt{Fichier|Changer le r�pertoire courant...}. Consulter
      aussi la \emph{R for Windows FAQ}.
    \end{itemize}
  \end{itemize}
\end{slide}

\begin{slide}
  \section{Consulter l'aide en ligne}
  \label{presentation:aide}

  Les rubriques d'aide des diverses fonctions disponibles dans S-Plus
  et \textsf{R} contiennent une foule d'informations ainsi que des
  exemples d'utilisation. Leur consultation est tout � fait
  essentielle.

  \begin{itemize}
  \item Pour consulter la rubrique d'aide de la fonction \code{foo},
    on peut entrer � la ligne de commande
\begin{Schunk}
\begin{Sinput}
> ?foo
\end{Sinput}
\end{Schunk}
  \item Dans Emacs, \code{C-c C-v foo RET}\indexess{C-c C-v} ouvrira
    la rubrique d'aide de la fonction \code{foo} dans un nouveau
    \emph{buffer}.
  \item Il existe plusieurs touches de raccourcis utiles pour la
    consultation des rubriques d'aide (voir la carte de r�f�rence
    ESS).
  \item Entre autres, la touche \texttt{l} permet d'ex�cuter ligne par
    ligne les exemples se trouvant � la fin de chaque rubrique d'aide.
  \end{itemize}
\end{slide}

%%% Local Variables:
%%% mode: latex
%%% TeX-master: "introduction_programmation_S_slides"
%%% End:

%\include{bases_slides}
%\include{operateurs_slides}
%\include{exemples_slides}
%\include{fonctions_slides}
%\include{avance_slides}

\include{ess_slides}

\end{document}

%%% Local Variables: 
%%% mode: latex
%%% TeX-master: t
%%% End: 
